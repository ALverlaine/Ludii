\chapter{Defines}  \label{Chapter:Defines}

The {\tt define} is a mechanism for replacing text in game descriptions with simple short labels, much like {\it macros} are used in programming languages. 
Defines are the first of several Ludii metalanguage features intended to make game descriptions clearer and more powerful.

\phantom{}
\noindent
Defines are useful for:
\begin{itemize}
\item Simplifying game descriptions by wrapping complex ludeme structures into simple labels.
\item Giving meaningful names to useful game concepts.
\item Gathering repeated ludeme structures that are duplicated across multiple games into a single reusable definition. 
\end{itemize}

\phantom{}
\noindent
Defines can occur anywhere in the game description file -- even within ludemes -- but are typically at the top of the file, before the {\tt game} reference, to impart some structure. 
Defines can be nested within other defines... but not themselves! 
The convention is to name each {\tt define} with a single compound word in ``UpperCamelCase'' format.

%--------------------------------------------------------------------------------

\section{Example}

For example, the game description for Breakthrough contains the following {\tt define}:

{\tt
\begin{verbatim}
    (define "ReachedTarget" (in (lastTo) (region Mover)) )
\end{verbatim}
}

This {\tt define} wraps up the concept the current mover's last ``to'' move landing in the target region with the label ``ReachedTarget''. 
The game's {\it end} rule can then be simplified and clarified as follows:

{\tt
\begin{verbatim}
    (end (if ("ReachedTarget") (result Mover Win)))
\end{verbatim}
}

This concept is used in many games, all of which can reuse this {\tt define} to simplify and clarify their own descriptions.

%--------------------------------------------------------------------------------

\section{Parameters}

Ludii {\tt define}s can be parameterised for greater flexibility. 
The parameters to be passed in to a define can take the following form:

{\tt
\begin{verbatim}
    keyword
    (clause ...)
    name:keyword
    name:(clause ...)
\end{verbatim}
}

Parameters are matched to {\it insertion points} of the form $\#N$ within the define, where $N$ is the number of the parameter to be instantiated. 
For example, the following define:

{\tt
\begin{verbatim}
    (define "Outcome" (result #1 #2))
\end{verbatim}
}

\noindent
can be instantiated with any of the following calls:

{\tt
\begin{verbatim}
    ("Outcome" Mover Win)    
    ("Outcome" (next) Lose)  
    ("Outcome" All Draw)    
\end{verbatim}
}

\noindent
to give:

{\tt
\begin{verbatim}
    (result Mover Win)    
    (result (next) Lose)  
    (result All Draw)    
\end{verbatim}
}

Parameterised {\tt define}s {\it must} be surrounded by brackets that enclose the label and its parameters when instantiated. 
Defines can contain arbitrary text, but should have balanced brackets, i.e. the same number of open and close brackets `(' for `)' and `\{' for `\}'. 
Non-parameterised {\tt define}s do not need to be bracketed, but it is recommended to do so for consistency and readability. 
Both of the following formats are allowed but the first format is preferred:

{\tt
\begin{verbatim}
    (end (if ("ReachedTarget") (result Mover Win)))
    (end (if  "ReachedTarget"  (result Mover Win)))
\end{verbatim}
}

%--------------------------------------------------------------------------------

\section{Null Parameters}

Sometimes a {\tt define} might be useful but its parameters do not match the current circumstance. 
In this case, a null parameter placeholder character '~' may be passed instead of that parameter, which simply instantiates to nothing.
Null parameters make {\tt define}s even more powerful, by allowing the same {\tt define} to be used in different ways by different games.
For example, the following {\tt define}:

{\tt
\begin{verbatim}
    (define "HopSequenceCapture" 
        (hop
            (between #1 #2 
                if:(isEnemy (who at:(between))) 
                (apply (remove (between) #3))
            )
            (to if:(in (to) (empty)))
            (consequence 
                (if (canMove 
                        (hop 
                            (from (lastTo)) 
                            (between #1 #2
                                if:(and (not (in (between) (sitesToClear))) 
                                   (isEnemy (who at:(between))))
                                (apply (remove (between) #3))
                            )
                            (to if:(in (to) (empty)))
                        )
                    ) 
                    (moveAgain)
                )
            )   
        )  
    )  
\end{verbatim}
}

\noindent
is called as follows in the game Coyote:
 
{\tt
\begin{verbatim}
    ("HopSequenceCapture" ~ ~ at:EndOfTurn) 
\end{verbatim}
}

The first two parameters are null placeholders, so all occurrences of ``\#1'' and ``\#2'' in the ``HopSequenceCapture'' {\tt define} will be instantiated with the empty string ``'', while all occurrences of ``\#3'' will be instantiated with ``at:EndOfTurn''.

%--------------------------------------------------------------------------------

\section{Known Defines}

External {\tt define}s called {\it known defines} can also be called within a game description simply by invoking their names (with suitable parameters). 
Each such known {\tt define} must: 
\begin{itemize}
\item be declared in a file with the same name as its label, 
\item have the file extension *.def, and 
\item be located in Ludii's "def" folder (or below it). 
\end{itemize}

\phantom{}
\noindent
The list of known defines provided with the Ludii distribution is given in Appendix~\ref{Appendix:KnownDefines}. \\
In addition, each game will typically have a known {\tt ai} metadata entry of the form:

{\tt
\begin{verbatim}
    (metadata
        ...
        (ai 
            "Chess_ai"
        )
    )    
\end{verbatim}
}

This is a reference to the known {\tt ai} define that is automatically generated for each game, which stores the relevant AI settings for that game and its various options.
Details of the  {\tt ai} metadata format are given in Chapter~\ref{Chapter:AIMetadata}.

