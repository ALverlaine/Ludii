\chapter{Options}  \label{Chapter:Options}

For many games there exist alternative rule sets and other variable aspects, such as different board sizes, number of pieces, starting positions, and so on. 
The Ludii game compiler supports an {\it option} mechanism to allow such alternatives for a game to be defined in a single description, to avoid the need to implement each one in its own file. 
Options are defined outside the main {\tt game} ludeme but typically used within it. 
Options are typically declared directly below the {\tt game ...)} ludeme, for clarity.

Options are instantiated at compile time and can be arbitrarily large, including choices between complete game descriptions if desired. 
Multiple options can be specified in combination, to give dozens or even hundreds of variant rule sets for a single game description.

%-------------------------------------------------------------------------------------

\section{Syntax}

Each set of options is declared with the {\tt option} keyword and constitutes an option {\it category} with a number of option {\it items}. 
Each option item has one or more named {\it arguments}.  
Option are described as follows:

{\tt
\begin{verbatim}
    (option "Heading <Tag> args:{ <argA> <argB> <argC> ... } {
        (item "Item X" <Xa> <Xb> <Xc> ... "Description of item X.")  
        (item "Item Y" <Ya> <Yb> <Yc> ... "Description of item Y.")****   
        (item "Item Z" <Za> <Zb> <Zc> ... "Description of item Z.")*   
        ...
    })
\end{verbatim}
}

\noindent
where:
\begin{itemize}
\item {\tt "Category A"} is the category name.
\item {\tt Tag} is a tag used to locate the position in the game description where the option is to be instantiated.
\item {\tt argA/B/C} are named arguments for each item.
\item {\tt "Item X/Y/Z"} are the item names.
\item {\tt Xa, Xb, Xc} are the actual option arguments to instantiate.
\item {\tt "Description of item X/Y/Z."} describe each item in user friendly terms.
\end{itemize}

\phantom{}
\noindent
Option items are referenced in the game description by tag-argument pairs {\tt <Tag:arg>}. \\
For example, this option call in the game description:

{\tt
\begin{verbatim}
    (ludeme <Tag:argB>)
\end{verbatim}
}

\noindent
would be instantiated as follows if the user selects the menu item ``Category A/Item Y":

{\tt
\begin{verbatim}
    (ludeme Yb)
\end{verbatim}
}

%-------------------------------------------------------------------------------------

\section{Option Priority}

A number of asterisks may optionally be appended to the end of each option item. 
The number of asterisks indicate that item's {\it priority} rating, with a higher number meaning higher priority. 

If no user-selected options are specified when a game is compiled, then the highest priority item within each category becomes the current option for that category. 
If more than one item exists with the highest priority rating, then the first item listed with this rating is chosen. 
For example, ``Item Y" would be the highest priority item for ``Category A'', with priority rating ****, and the default option to be instantiated in the absence of any user-selected options.

%-------------------------------------------------------------------------------------

\section{Example}

The following example shows the option mechanism in action for the game of Hex. 
The game description assumes the existence of two option categories -- {\tt Board} and {\tt Result} -- with item arguments {\tt size} and {\tt type}, respectively.  

{\tt
\begin{verbatim}
    (game "Hex"  
        (players 2)  
        (equipment { 
            (board (hex Diamond <Board:size>)) 
            (piece "Ball" Each)
            (regions P1 { (sites Side NE) (sites Side SW) })
            (regions P2 { (sites Side NW) (sites Side SE) })
        })  
        (rules 
            (meta (swap))
            (play (move Add (to (sites Empty)))) 
            (end (if (is Connected Mover) (result Mover <Result:type>))) 
        )
    )
\end{verbatim}
}

\noindent
The two option categories are declared as follows. 
Note that the 11x11 option has the highest priority, while the 10x10, 14x14 and 17x17 options are next priority below. 
These are the most common sizes of Hex boards, so are the most interesting options for the user.

{\tt
\begin{verbatim}
    (option "Board Size" <Board> args:{ <size> } {
        (item   "3x3"  <3> "The game is played on a 3x3 board.")   
        (item   "4x4"  <4> "The game is played on a 4x4 board.")   
        (item   "5x5"  <5> "The game is played on a 5x5 board.")   
        (item   "6x6"  <6> "The game is played on a 6x6 board.")   
        (item   "7x7"  <7> "The game is played on a 7x7 board.")   
        (item   "8x8"  <8> "The game is played on a 8x8 board.")   
        (item   "9x9"  <9> "The game is played on a 9x9 board.")   
        (item "10x10" <10> "The game is played on a 10x10 board.")*   
        (item "11x11" <11> "The game is played on a 11x11 board.")****   
        (item "12x12" <12> "The game is played on a 12x12 board.")   
        (item "13x13" <13> "The game is played on a 13x13 board.")   
        (item "14x14" <14> "The game is played on a 14x14 board.")*   
        (item "15x15" <15> "The game is played on a 15x15 board.")   
        (item "16x16" <16> "The game is played on a 16x16 board.")   
        (item "17x17" <17> "The game is played on a 17x17 board.")*   
        (item "18x18" <18> "The game is played on a 18x18 board.")   
        (item "19x19" <19> "The game is played on a 19x19 board.")   
    })

    (option "End Rules" <Result> args:{ <type> } {
        (item "Standard"  <Win>  "The first player to connect his two sides wins.")*   
        (item "Misere"    <Loss> "The first player to connect his two sides loses.")   
    })
\end{verbatim}
}

\noindent
If the user selects the ``Board Size/9x9" and ``End Rules $>$ Misere'' menu item, then the game will be instantiated as follows during compilation, to give a {\it mis\`{e}re} version of the game on a 9$\times$9 board:

{\tt
\begin{verbatim}
    (game "Hex"  
        (players 2)  
        (equipment { 
            (board (rhombus 9)) 
            (piece "Ball" Each)
            (regions P1 { (sites Side NE) (sites Side SW) })
            (regions P2 { (sites Side NW) (sites Side SE) })
        })  
        (rules 
            (meta (swap) )
            (play (add (empty))) 
            (end (if (isConnected Mover) (result Mover Loss))) 
        )
    )
\end{verbatim}
}

